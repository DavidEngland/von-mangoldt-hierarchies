% filepath: /Users/davidengland/Documents/GitHub/generalized-factorials-stirling/Analytic_Hierarchies_Stieltjes.tex
\documentclass[11pt]{article}
\usepackage[utf8]{inputenc}
\usepackage{amsmath,amssymb,amsfonts}
\usepackage{hyperref}
\usepackage{geometry}
\geometry{margin=1in}
\usepackage{enumitem}
\usepackage{microtype}
\usepackage{mathtools}

\title{Analytic Hierarchies of the von Mangoldt Function and the Stieltjes Constants}
\author{}
\date{}

\newcommand{\LM}{\Lambda}
\newcommand{\z}{\zeta}

\begin{document}
\maketitle

\section*{Overview}
Level: Advanced Graduate (Analytic Number Theory). Prerequisites: Dirichlet series and Euler products; basic complex analysis; Mellin transforms; elementary prime number theory.

\section{Lecture 1 --- The von Mangoldt Function and Its Generating Series}
\subsection{The von Mangoldt function}
\[
\LM(n)=
\begin{cases}
\log p,& n=p^k\ (p\ \text{prime},\ k\ge1),\\[4pt]
0,&\text{otherwise}.
\end{cases}
\]

\subsection{Dirichlet generating function}
For $\Re(s)>1$,
\[
\sum_{n\ge1}\frac{\LM(n)}{n^s}=-\frac{\z'(s)}{\z(s)}.
\]

\paragraph{Commentary.} This identity connects primes and analytic structure; zeros of $\z(s)$ induce oscillations in prime counts.

\paragraph{Exercises.}
\begin{enumerate}
\item From $\z(s)=\prod_p(1-p^{-s})^{-1}$ show $-\z'(s)/\z(s)=\sum_p\frac{\log p}{p^s-1}$ and expand to recover $\sum_n \LM(n)/n^s$.
\item Use the logarithmic derivative to show primes dominate the behaviour of $\z(s)$ near $s=1$.
\item Prove $\Psi(x)=\sum_{n\le x}\LM(n)$ has the same asymptotic order as $x$.
\end{enumerate}

\section{Lecture 2 --- The von Mangoldt Hierarchies}
For integer $k\ge0$ define
\[
\sum_{n\ge1}\frac{\LM_k(n)}{n^s}=(-1)^k\frac{d^k}{ds^k}\!\Big[-\frac{\z'(s)}{\z(s)}\Big],\qquad \Re(s)>1.
\]
\[
\Psi_k(x)=\sum_{n\le x}\LM_k(n).
\]

\paragraph{Exercises.}
\begin{enumerate}
\item Verify $\LM_0(n)=\LM(n)$ and $\LM_1(n)=-(\log n)\LM(n)$ and check the Dirichlet series.
\item Tabulate $\LM_k(n)$ for small $n$ with a short script.
\item Interpret $\LM_k(n)$ probabilistically (cumulant viewpoint).
\end{enumerate}

\section{Lecture 3 --- Mellin Transforms and Explicit Formulas}
\[
\int_0^\infty \Psi_k(x)x^{-s-1}\,dx=\frac{1}{s}\sum_{n\ge1}\frac{\LM_k(n)}{n^s}
=\frac{(-1)^k}{s}\frac{d^k}{ds^k}\!\Big[-\frac{\z'(s)}{\z(s)}\Big].
\]
Mellin inversion ($c>1$):
\[
\Psi_k(x)=\frac{1}{2\pi i}\int_{c-i\infty}^{c+i\infty}\frac{(-1)^k}{s}\frac{d^k}{ds^k}\!\Big[-\frac{\z'(s)}{\z(s)}\Big]x^s\,ds.
\]

\paragraph{Commentary.} Shift the contour left to pick up residue at $s=1$ (main term) and contributions from nontrivial zeros $\rho$ (oscillatory terms).

\paragraph{Exercises.} Partial summation verification, contour shift residue computations, zero-contribution derivations, compute main term for $\Psi_1(x)$ via Stieltjes constants.

\section{Lecture 4 --- The Stieltjes Constants and Their Generating Function}
The Laurent expansion of $\zeta$ at $s=1$:
\[
\zeta(1+u)=\frac{1}{u}+\sum_{n=0}^\infty\frac{(-1)^n\gamma_n}{n!}u^n.
\]
The exponential generating function (EGF):
\[
G(t)=\sum_{n=0}^\infty\frac{\gamma_n}{n!}t^n=\zeta(1-t)+\frac{1}{t}.
\]

\paragraph{Commentary.} $G(t)$ is entire (singularity cancels at $t=0$); singularities occur at $t=1-\rho$ where $\rho$ runs over zeros of $\z(s)$, so the behaviour of $\gamma_n$ reflects the zero distribution.

\paragraph{Exercises.} Verify the EGF identity, determine radius of convergence, compute small $\gamma_n$ numerically.

\section{Lecture 5 --- Relating Stieltjes Constants to Prime Fluctuations}
Local expansion:
\[
-\frac{\z'(s)}{\z(s)}=\frac{1}{s-1}-\sum_{n=0}^\infty \eta_n (s-1)^n,
\]
with $\eta_n$ polynomially expressible in the $\gamma_m$. Thus $\LM_k(n)$ relates to derivatives at $s=1$ and to the $\eta_k$.

\paragraph{Exercises.} Derive $\eta_n$ for small $n$, write explicit formulas for $\Psi_k(x)$ up to $k=2$, show contribution of zeros as $x^\rho P_k(\rho,\log x)$.

\section{Lecture 6 --- Further Insights on the Stieltjes Constants}
\subsection*{Multiple useful viewpoints}
\begin{itemize}
\item \textbf{Cumulant/Probabilistic:} $\gamma_n$ viewed as cumulant-like coefficients from $G(t)$; suggests smoothing transforms to study sign patterns.
\item \textbf{Spectral/Entire-function:} Singularities of $G$ at $t=1-\rho$ make $\{\gamma_n\}$ sensitive to zeros near $s=1$.
\item \textbf{Asymptotic/Saddle-point:} Saddle-point analysis of $G(t)$ predicts factorial-scale growth of $\gamma_n$ with oscillatory modulation.
\end{itemize}

\subsection*{Exact classical representation}
\[
\gamma_n=\lim_{m\to\infty}\Big(\sum_{k=1}^m\frac{(\log k)^n}{k}-\frac{(\log m)^{n+1}}{n+1}\Big).
\]

\paragraph{Experiments.} Plot scaled values $\gamma_n/n!$, compute zero-contribution truncations, apply Borel or Bell-polynomial smoothing.

\section{Lecture 7 --- Open Problems and Research Directions}
\begin{itemize}[leftmargin=*]
\item Analytic: uniform asymptotics for $\Psi_k(x)$ in $k$.
\item Probabilistic: random model with cumulants equal to $\gamma_n$.
\item Positivity: transforms yielding positive sequences from $\gamma_n$.
\item Spectral: rigorous spectral interpretation of $G(t)=\zeta(1-t)+1/t$.
\item Computational: stable computation of $\Psi_k(x)$ and derivatives of $\zeta$.
\end{itemize}

\section*{Projects and Further Reading}
Projects: numerical zero-sum experiments; saddle-point asymptotics for $\gamma_n$; symbolic Bell-polynomial relations; probabilistic prime-process models.

References include Titchmarsh, Coffey, Knessl \& Coffey, Edwards, Montgomery \& Vaughan.

\bigskip
\noindent\textbf{Compilation note.} Run \texttt{pdflatex} or \texttt{lualatex} twice to resolve cross-references.

\end{document}